
\newpage

\section{伤害系统与一般性魔能力搭配}

游戏基本伤害公式为:

伤害=(攻击能力值*技能倍率*(1+攻击补正)-防御力*(1+防御补正)/2)*(1+基本伤害)*武器抗性补正*元素抗性补正*魔法范围补正*暴击补正*连击补正.

攻击补正和防御补正可以按R2看到,游戏给的伤害补正为计算基本伤害/武器抗性/元素抗性/魔法范围补正之后减1的数值,不含暴击伤害.

攻击魔法的判定值是INT,部分固有技能是特定能力如重骑士的第二个固有技能就是DEF,其他技能均依赖于武器.

有随机浮动.

\begin{enumerate}

	\item 基本伤害补正

	如[冲锋攻击],XX则增加XX\%伤害.

	不同基本伤害补正为叠加效果.

	\item 武器抗性补正

	伤害调整为(1-对应武器抗性/100)倍,魔法不计算武器抗性.

	\item 元素抗性补正
	
	属性伤害调整为(1-对应属性抗性/100)倍,最大1.99倍.

	\item 魔法范围补正
	
	负补正,范围每增加1总伤害降低10\%,3*3的攻击魔法效果只有单体的20\%.

	\item 暴击伤害补正

	暴击住人是增加基本暴击数值.100暴击住人的结果是暴击变成2.5倍(没住人是1.5倍).

	暴击魔能力的效果是叠乘,它们增加的都是总伤害,在攻击力和基本伤害有一定数值后暴击伤害成为最强的攻击属性.而相应的,[安全防护]也变成最强的防御能力.

	[神风]\{杰洛肯\}是SPD高于对手加100\%.

	[炼狱]\{黑暗骑士\}是无条件50\%.

	[暴击点]\{佩妲\}是邻接无敌人加30\%.

	需要注意的是[赌徒]\{角色界\}实际为50\%几率暴击伤害+50\%(并不是它自己说的100\%),还50\%几率伤害变为0.

	赌徒的存在导致最大伤害几乎不可能达成,因为那需要触发10次50\%几率,也就是1/1024的几率,而每次准备时间都要差不多半小时.

	\item 连击补正
	
	连击补正在20连击后达到上限3.0倍.

	\item 技能倍率

	E级的技能是1.00 $\rightarrow$ 1.07倍率,每提升一档倍率加0.07.

	如E技能到C+,E$\rightarrow$E+$\rightarrow$D$\rightarrow$D+$\rightarrow$C$\rightarrow$C+提升了5档,所以C+技能的倍率在1.35$\rightarrow$1.42.

	从这里也可以知道这代的技能强化效果远远弱了三代四代的指数爆炸效果.

	SS+技能强化后仍然会提升倍率.

\end{enumerate}

{\color{red}如果想提高对游戏系统的认知,多按R2并问自己为什么是这个数字.}

\newpage

有了这个具体伤害公式,搭配魔能力就有了方向可依.

首先由于能力值可以通过队友/部队效果/buff之类的增加很多,所以要尽量压缩能力类魔能力的使用,最多最多不要超过4个泛用格子.前魔帝的威猛暴力花费6个格子去拿50\%能力永远不要用.

最优先的魔能力当然是直接影响总倍率的,一个因素就能影响总伤害,也就是公式里的元素抗性/武器抗性/暴击伤害/连击倍率.然而敌方武器抗性不可改变,元素抗性改变可以用队友做到,连击倍率是队友配合达成.所以自身的魔能力最优先能力就变成暴击伤害,100暴击住人不必说,炼狱必不可少.能触发神风和暴击点就必须带上,均无法达到才考虑换成别的.为了避免魔法的自身削减效果,魔法需要尽可能使用低范围的,最佳的当然是单体.

刷道具界/道具神或者其他一般不具备免疫能力下降的敌人自然是神风+炼狱+暴击点,所以实际上一般情况下你只有10个格子可以选.

剩下10个格子可以装攻击力或者伤害值,然而也就玛娜枯竭/拼死/舍身/无偿的爱这些.泛用格子加伤害的太少而且数值低,一般向战斗力只要一个冲锋攻击就行了,如果重要战斗可以考虑用女仆重置移动达到直接200\%的伤害上限.

