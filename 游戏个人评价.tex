\newpage
\section{游戏个人评价}

	看过中国的贴吧论坛,也翻过北美的交流社区,粉丝聚集地确实对这作评价很高,对于如此差的销量(对比前代诸作品)感到疑惑.

	这里就来谈谈五代的问题到底出在哪里,人作死就会死,不要怪玩家抛弃.

	不用心之关卡设计.虽然它以刷刷刷著称,但是历代作品在通关前都是一个完整的SLG作品.吧里不少人从这代开始玩可能没感觉,我从一代玩到现在五代,通关日版之后对本作的剧情战关卡设计评价是不及格.除了初期的几话谈得上策略性(2-4的绕后开铳,2-5的坦克卡位,3话利用地形反客为主),后面的剧情战纯属无脑堆怪.就是上去砍,不是他死就是你死.导致无双玩家可能喜欢,而策略流玩家则玩着就难受.想想历代作品就算没有里世界光是剧情关卡也有很多地图需要动下脑子才能过,这代剧情战真是相当无脑.即便前作玩后过的时间相当久了随口举几个例子还是没问题的:四代的前期远程隔墙卡位击破,人塔+投掷传送水晶打破敌人的包围阵型,用弓箭范围卡敌人铳范围,用铳卡敌人弓箭范围.随便拿几个关卡都完爆这作.更别说还有专门用来挑战的数十个里世界关卡.这代的挑战关量少,质量也就里世界的一般水准.

	不用心之水晶系统.游戏的重要组成部分水晶系统在这代几乎毫无存在感.水晶系统是用来简化替代其他SLG作品的地形效果.历代中后期剧情战都会有一些恶心的水晶,比如无敌/伤害反转/敌超强化/沉默/禁止远程/禁止邻接什么的,这代这些特殊水晶总共出现了几次?哟,这些水晶会给无脑打怪的玩家带来麻烦,于是他们就把整个水晶系统都无视了.水晶唯一的存在感大约是在那个奖杯里.

	坑钱之强行"取消后日谈".玩过之前作品的玩家大概会知道大部分前代主要角色都会在后日谈剧情作为BOSS出现,击败后入队.只有那些边缘角色/其他作品角色才会作为DLC出售.而这代强行全部前代角色都作为DLC必须购买才能用,于是一方面玩家必须花钱买前作角色,另外一方面玩家可选的角色其实变少了,因为这些前代主角占了DLC的位置.如此倒行逆施逼迫玩家花钱买前主角无异于饮鸩止渴,会让玩家寒心,

	当然优点也是不少的,比如角色自定义性更强了,技能强化更平衡了,SP这个数值终于有了意义,后期的普攻终于有用了,后期终于不会被秒了.然而这些对于一部分只通剧情的玩家是不存在的---根据我在国外论坛的经历这部分玩家其实不少,人家就是去玩搞笑剧情和一个单纯的SLG的.

	个人而言,对六的状况持悲观状态.


	(6真的出了,但是悲观确实是对的)