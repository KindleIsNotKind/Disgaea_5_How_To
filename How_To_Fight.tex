
\newpage

\section{一般向战术介绍与进阶}

日版刚发售时DLC是分拨发放,会导致一些奇葩的战术无法使用.本篇主要包含一些特殊过关方法,多数与DLC有关.

	\subsection{传统反击}

	主要目的是单体高HP强敌击破.

	如魔能力搭配一章所述,单体强敌击破最简单易行方法,小地图群体击破简单方法.

	广为人知但简单粗暴,强行加入这里.

	最合适的反击角色是男格斗家,给自己上个必定反击的buff,而后女仆重置行动,格斗家去攻击目标即可.只需要带[强制反击]就能完全必定20反击,相比一般角色可以节省大量泛用能力格([迎击架势+防御态势]).

	如果有DLC可以考虑带上普莉耶的[强反击],能力弱HP低的时候带上[对打练习].

	\subsection{魔法}

	这作的魔法是个很极端的存在,单体很强很强,群体则弱到不能玩.

	单体魔法有历代游戏最远的攻击距离,还有各种增加魔法射程的魔能力,一些是自身魔能力,一些是队友提供的,另外则是二代DLC的女主魔能力由部队队员提供.

	群体魔法毫无作为,范围每增加1总伤害就减少10个百分点,单体是100\%伤害,两格是90\%,到3*3只剩20\%了.没有适用性.不能刷练武.概括的话就是:渣.

	单体魔法主要有两个玩法,一个以[双头]为核心的原地炮台,一个是以[冲锋攻击]为核心的移动法师.然而无论哪种流派[元素之力]+[加速射击]+\{精神集中\}都属于不可替代的能力,可选[武士道],推荐\{鹰眼\}+西瓜屋,保证必中的同时增加50\%伤害.移动法师配合LV99的魔法射程与极限距离的冲锋/加速与元素弱点伤害可以获得方便的瞬间爆发,不做任何准备且面对未知的敌人是最可靠的伤害手段.适用于道具界小屋挑战,大地图风筝敌人,复杂地形远距离打击等情况.中后期配合贤者的双发可以秒掉近三百亿HP的道具神.

	当然要使用[元素之力]是少不了99的抗性的,原始抗性只要高于-50就可以利用绿住人彩虹屋+抗性屋达到99的上限,所以法师流派对于初始职业几乎没有限制,因为抗性为-75的职业本身就很少.

	如果敌人抗性较高可以考虑找上带了[反元素]的辅助站到敌人旁边.

	\subsection{奥义COMBO}

	这个战术是属于DLC玩家的大规模杀伤性武器,平常还是少用保持封印状态.打巴尔和极限伤害再解封吧.

	需要拉哈尔四人组DLC,拉哈尔酱,风花三人组DLC,幽灵王国三人组DLC.

	佩妲奥义->艾多娜奥义->拉哈尔奥义->艾多娜奥义->拉哈尔奥义->玛裘丽特奥义->拉哈尔酱奥义->玛裘丽特奥义->拉哈尔酱奥义

	瞬间叠满拉哈尔的固有能力+100\%能力,玛裘丽特的固有能力+200\%攻击力,战斗热情+200\%攻击力,死神的行进+200\%伤害值,地图我方魔物全体获得额外200\%攻击力.

	\newpage
	
	\subsection{钓鱼台}

	最终战术.

	时过境迁,以往作品的无敌举人大法不仅被破还被敌人学了去,好在钓鱼台大法还没被学走.

	钓鱼台是指利用技能高度差的策略,很多人在主线2-4曾经被那个高冷的女魔法师欺负过,那关其实就是游戏的高度差教学.

	游戏里的[盗贼]是能强行改变地形高度差的职业,可以造箱子造炸药桶.在一些凹凸不平的关卡(银行五壮士/修罗开启关)改变高度就可以造成敌人技能容差不够的结果,无法普攻你无法技能打到你,于是玩家处于无敌状态,只要利用一些高容差技能(主要指魔法因为其他技能都有范围限定)就可以慢慢磨死.

	钓鱼台战术的巅峰则是来自于最后一弹DLC,泽塔三人组.

	泽塔有着人造地形的能力,配合女儿的两次奥义,可以达到绝大多数敌人根本摸不到的地方,可以说除了最后的巴尔关泽塔三人组等于绝对无敌的存在.

	原因:一般人物的固有技/武器技/魔法最大容差也只有48/64,两次造塔已经高出了所有这些技能的攻击范围.

	游戏里存在的99/99技能其实是无视高度技能,即便真的高于99也仍然会被打到,包含贤者的全屏,女天使的自爆,以及巴尔的技能.但是,但是,贤者的全屏会攻击自己人所以游戏AI设定敌人贤者不会放全屏.而女天使是DLC除了道具界根本看不到,这自爆技能也不会放.剩下唯一可以打到双塔上三人组的只有巴尔.

	而我方贤者和三人组的预言家(主魔能力全屏随机攻击敌人)则是有无视高度技能的.所以结果就是父女造双塔,人多贤者开炮轰,轰到人少了就白毛上来一直结束回合等胜利.

	除了巴尔关,其他地方父女同时满复仇槽等于无敌.

	全程20星玩法如果打不过挑战关可以用这个作弊...

	我当初就是2000万能力预言家用这个打死的修罗二十星邪帝.
